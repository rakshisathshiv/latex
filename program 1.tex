\documentclass{article} 
\usepackage{authoraftertitle}
\usepackage{fancyhdr} 
\fancyhead{} 
\fancyhead[L]{left header} 
\fancyhead[R]{right header \quad \thepage} 
\fancyfoot{} 
\fancypagestyle{plain}{   
\fancyhead{}   
\fancyhead[C]{Intent based Search Diversification}   
\fancyfoot{}   
\fancyfoot[C]{CIT, TUMAKURU} } 
\pagestyle{fancy} 
\title{Search Optimization Process} 
\begin{document} 
\maketitle
\section{Context based Diversification} The  human  era  is  evolved  and  dominated  through  the  ultimate  intention  to  know about the Universe and its assets more and more. This in turn persuaded him to gather the immense information of need in the form of theory, tools, intuitions, visuals, and  ultimately as the form of abstract objects.
\section{ Conceptualization of the Search Queries} Deciding the context of the search query based on its representation over a concept network using fuzzy methods provides a better thrust to the overall search process. The existing context based search diversification process emphasizes the importance of the numerical representation of the query over a data repository. The search operation can use these semantically meaningful segments as a confident segment in the conceptual network.
\end{document}
